\documentclass[UTF8]{ctexart}
\usepackage{graphicx}
\graphicspath{{./images/}}
\usepackage{amsmath,amsfonts}
\usepackage{enumerate}
\DeclareMathOperator{\exgcd}{exgcd}
    \title{中国剩余定理学习笔记}
    \author{beck}
    \date{\today}s
\begin{document}
\maketitle
\section*{形式描述}
中国剩余定理,又称中国余数定理,是数论中的一个关于一元线性同余方程组的定理,说明了一元线性同余方程组有解的准则以及求解方法。也称为孙子定理,古有“韩信点兵”、“孙子定理”、“求一术”(宋沈括)、“鬼谷算”(宋周密)、“隔墙算”(宋 周密)、“剪管术”(宋杨辉)、“秦王暗点兵”、“物不知数”之名。
用现代数学的语言来说明的话,中国剩余定理给出了以下的一元线性同余方程组:
\[
    (S):\quad
    \begin{cases}
        \begin{aligned}
            x & \equiv a_1 \;(\bmod\;m_1) \\
            x & \equiv a_2 \;(\bmod\;m_2) \\
            & \vdots \\
            x & \equiv a_n \;(\bmod\;m_n)
        \end{aligned}
    \end{cases}
\]
有解的判定条件,并用构造法给出了在有解情况下解的具体形式。
中国剩余定理说明:假设整数$m_1,m_2,\ldots,m_n$其中任两数互质,则对任意的整数:$a_1,a_2,\ldots,a_n$,方程组$(S)$有解,并且通解可以用如下方式构造得到:
\begin{enumerate}[1.]
    \item 设$M = m_1 \times m_2 \times \ldots \times m_n = {\displaystyle \prod_{i=1}^n m_i}$是整数$m_1,m_2,\ldots,m_n$的乘积,并设$M_i=M/m_i$,$\forall i \in \left\{1,2,\ldots,n\right\}$,即$M_i$是除了$m_i$以外的$n - 1$个整数的乘积。
    \item 设$t_i = M_i^{-1}$为$M_i$模$m_i$的数论倒数:$t_iM_i \equiv 1\;(\bmod\;m_i)$,$\forall i \in \left\{1,2,\ldots,n\right\}.$
    \item 方程组$(S)$的通解形式为:$x = a_1t_1M_1 + a_2t_2M_2 + \ldots + a_nt_nM_n + kM = kM + {\displaystyle \sum_{i=1}^n a_it_iM_i}$,$k \in \mathbb{Z}.$在模$M$的意义下,方程组$(S)$只有一个解:$x = {\displaystyle \sum_{i=1}^n a_it_iM_i}.$
\end{enumerate}
\subsection*{证明}
从假设可知,对任何$i \in \left\{1,2,\ldots,n\right\}$,由于$\forall j \in \left\{1,2,\ldots,n\right\}$,$j \neq i$,$\gcd(m_i,m_j) = 1$,所以$\gcd(m_i,M_i) = 1.$这说明存在整数$t_i$使得$t_iM_i \equiv 1 \;(\bmod\;m_i).$这样的$t_i$叫做$M_i$模$m_i$的数论倒数。考察乘积$a_it_iM_i$可知:
\[
\begin{aligned}
    & a_it_iM_i \equiv a_i \cdot 1 = a_i \;(\bmod\;m_i), \\
    & \forall j \in \left\{1,2,\ldots,n\right\}, j \neq i, a_jt_jM_j \equiv 0 \;(\bmod\;m_i).
\end{aligned}
\]
所以$x = a_1t_1M_1 + a_2t_2M_2 + \ldots + a_nt_nM_n$满足:
\[ \forall i \in \left\{1,2,\ldots,n\right\}, x = a_it_iM_i + {\displaystyle \sum_{j \neq i} a_jt_jM_j} \equiv a_i + {\displaystyle \sum_{j \neq i} 0 = a_i \;(\bmod\;m_i).} \]
这说明$x$就是方程组$(S)$的一个解。
另外,假设$x_1$和$x_2$都是方程组$(S)$的解,那么:
\[ \forall i \in \left\{1,2,\ldots,n\right\}, x_1 - x_2 \equiv 0 \;(\bmod\;m_i). \]
而$m_1,m_2,\ldots,m_n$两两互质,这说明$M = {\displaystyle \prod_{i=1}^n m_i}$整除$x_1 - x_2.$所以方程组$(S)$的任何两个解之间必然相差$M$的整数倍。而另一方面,$x = a_1t_1M_1 + a_2t_2M_2 + \ldots + a_nt_nM_n$是一个解,同时所有形式为:
\[ a_1t_1M_1 + a_2t_2M_2 + \ldots + a_nt_nM_n + kM = kM + {\displaystyle \sum_{i=1}^n a_it_iM_i,\; k \in \mathbb{Z}} \]
的整数也是方程组$(S)$的解。所以方程组$(S)$所有的解的集合就是:
\[ \left\{kM + {\displaystyle \sum_{i=1}^n a_it_iM_i\;;\quad k \in \mathbb{Z}}\right\} \]
\section*{模反元素}
模反元素也称为模倒数,或者模逆元。
一整数$a$对同余之模反元素是指满足以下公式的整数$b$
\[ a^{-1} \equiv b \;(\bmod\;n). \]
也可以写成以下的式子
\[ ab \equiv 1 \;(\bmod\;n). \]
整数$a$对模数$n$之模反元素存在的充分必要条件是$a$和$n$互质,若此模反元素存在,在模数$n$下的除法可以用和对应模反元素的乘法来达成,此概念和实数除法的概念相同。
\subsection*{求模反元素}
\subsubsection*{用扩展欧几里得算法}
设$\exgcd(a,n)$为扩展欧几里得算法的函数,则可得到$ax + ny = g$,$g$是$a,n$的最大公因数。
\paragraph*{若$g = 1$}~\\
则该模反元素存在,根据结果$ax + ny = 1$
\\ 在$\bmod\;n$之下,$ax + ny \equiv ax \equiv 1$,根据模反元素的定义$a \times a^{-1} \equiv 1$,此时$x$即为$a$关于模$n$的其中一个模反元素。
\\ 事实上,$x + kn(k \in \mathbb{Z})$都是$a$关于模$n$的模反元素,这里我们取最小的正整数解$x\;\bmod\;n\;(x < n)$。
\paragraph*{若$g \neq 1$}~\\
则该模反元素不存在。
\\ 因为根据结果$ax + ny \neq 1$,在$\;\bmod\;n\;$之下,$ax \equiv g (g < n)$不会同余于$1$,因此满足$a \times a^{-1} \equiv 1$的$a^{-1}$不存在。
\subsubsection*{用欧拉定理}
欧拉定理证明当$a,n$为两个互质的正整数时,则有$a^{\varphi(n)} \equiv 1 \;(\bmod\;n)$,其中$\varphi(n)$为欧拉函数(小于等于$n$且与$n$互质的正整数个数)。
\\ 上述结果可分解为$a^{\varphi(n)} = a \times a^{\varphi(n) - 1} \equiv 1 \;(\bmod\;n)$,其中$a^{\varphi(n) - 1}$即为$a$关于模$n$之模反元素。
\section*{辗转相除法}
在数学中,辗转相除法,又称欧几里得算法(英语:Euclidean algorithm),是求最大公约数的算法。辗转相除法首次出现于欧几里得的《几何原本》(第VII卷,命题i和ii)中,而在中国则可以追溯至东汉出现的《九章算术》。
\\ 两个整数的最大公约数是能够同时整除它们的最大的正整数。辗转相除法基于如下原理:两个整数的最大公约数等于其中较小的数和两数相除余数的最大公约数。例如,$252$和$105$的最大公约数是$21\;(\;252 = 21 \times 12;105 = 21 \times 5 \;)$;因为$252 − 105 = 21 × (12 − 5) = 147$,所以$147$和$105$的最大公约数也是$21$。在这个过程中,较大的数缩小了,所以继续进行同样的计算可以不断缩小这两个数直至其中一个变成零。这时,所剩下的还没有变成零的数就是两数的最大公约数。由辗转相除法也可以推出,两数的最大公约数可以用两数的整数倍相加来表示,如$21 = 5 × 105 + (−2) × 252$。这个重要的结论叫做贝祖定理。
\subsection*{算法描述}
\subsubsection*{计算过程}
\subsubsection*{正确性的证明}
\section*{扩展欧几里得算法}
\subsection*{算法和举例}
\subsection*{证明}
\section*{裴蜀定理}
\subsection*{证明}
\end{document}
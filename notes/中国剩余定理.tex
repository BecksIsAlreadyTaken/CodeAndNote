\documentclass[UTF8]{ctexart}
\usepackage{amsmath,amsfonts}
\usepackage{enumerate}
    \title{中国剩余定理学习笔记}
    \author{beck}
    \date{\today}
\begin{document}
\maketitle
\section{形式描述}
中国剩余定理,又称中国余数定理,是数论中的一个关于一元线性同余方程组的定理,说明了一元线性同余方程组有解的准则以及求解方法。也称为孙子定理,古有“韩信点兵”、“孙子定理”、“求一术”(宋沈括)、“鬼谷算”(宋周密)、“隔墙算”(宋 周密)、“剪管术”(宋杨辉)、“秦王暗点兵”、“物不知数”之名。
用现代数学的语言来说明的话,中国剩余定理给出了以下的一元线性同余方程组:
\[
    (S):\quad
    \begin{cases}
        \begin{aligned}
            x & \equiv a_1 \;(\bmod\;m_1) \\
            x & \equiv a_2 \;(\bmod\;m_2) \\
            & \vdots \\
            x & \equiv a_n \;(\bmod\;m_n)
        \end{aligned}
    \end{cases}
\]
有解的判定条件,并用构造法给出了在有解情况下解的具体形式。
中国剩余定理说明:假设整数$m_1,m_2,\ldots,m_n$其中任两数互质,则对任意的整数:$a_1,a_2,\ldots,a_n$,方程组$(S)$有解,并且通解可以用如下方式构造得到:
\begin{enumerate}[1.]
    \item 设$M = m_1 \times m_2 \times \ldots \times m_n = {\displaystyle \prod_{i=1}^n m_i}$是整数$m_1,m_2,\ldots,m_n$的乘积,并设$M_i=M/m_i$,$\forall i \in \left\{1,2,\ldots,n\right\}$,即$M_i$是除了$m_i$以外的$n - 1$个整数的乘积。
    \item 设$t_i = M_i^{-1}$为$M_i$模$m_i$的数论倒数:$t_iM_i \equiv 1\;(\bmod\;m_i)$,$\forall i \in \left\{1,2,\ldots,n\right\}.$
    \item 方程组$(S)$的通解形式为:$x = a_1t_1M_1 + a_2t_2M_2 + \ldots + a_nt_nM_n + kM = kM + {\displaystyle \sum_{i=1}^n a_it_iM_i}$,$k \in \mathbb{Z}.$在模$M$的意义下,方程组$(S)$只有一个解:$x = {\displaystyle \sum_{i=1}^n a_it_iM_i}.$
\end{enumerate}
\subsection{证明}
从假设可知,对任何$i \in \left\{1,2,\ldots,n\right\}$,由于$\forall j \in \left\{1,2,\ldots,n\right\}$,$j \neq i$,$\gcd(m_i,m_j) = 1$,所以$\gcd(m_i,M_i) = 1.$这说明存在整数$t_i$使得$t_iM_i \equiv 1 \;(\bmod\;m_i).$这样的$t_i$叫做$M_i$模$m_i$的数论倒数。考察乘积$a_it_iM_i$可知:
\[
\begin{aligned}
    & a_it_iM_i \equiv a_i \cdot 1 = a_i \;(\bmod\;m_i), \\
    & \forall j \in \left\{1,2,\ldots,n\right\}, j \neq i, a_jt_jM_j \equiv 0 \;(\bmod\;m_i).
\end{aligned}
\]
所以$x = a_1t_1M_1 + a_2t_2M_2 + \ldots + a_nt_nM_n$满足:
\[ \forall i \in \left\{1,2,\ldots,n\right\}, x = a_it_iM_i + {\displaystyle \sum_{j \neq i} a_jt_jM_j} \equiv a_i + {\displaystyle \sum_{j \neq i} 0 = a_i \;(\bmod\;m_i).} \]
这说明$x$就是方程组$(S)$的一个解。
另外,假设$x_1$和$x_2$都是方程组$(S)$的解,那么:
\[ \forall i \in \left\{1,2,\ldots,n\right\}, x_1 - x_2 \equiv 0 \;(\bmod\;m_i). \]
而$m_1,m_2,\ldots,m_n$两两互质,这说明$M = {\displaystyle \prod_{i=1}^n m_i}$整除$x_1 - x_2.$所以方程组$(S)$的任何两个解之间必然相差$M$的整数倍。而另一方面,$x = a_1t_1M_1 + a_2t_2M_2 + \ldots + a_nt_nM_n$是一个解,同时所有形式为:
\[ a_1t_1M_1 + a_2t_2M_2 + \ldots + a_nt_nM_n + kM = kM + {\displaystyle \sum_{i=1}^n a_it_iM_i,\; k \in \mathbb{Z}} \]
的整数也是方程组$(S)$的解。所以方程组$(S)$所有的解的集合就是:
\[ \left\{kM + {\displaystyle \sum_{i=1}^n a_it_iM_i\;;\quad k \in \mathbb{Z}}\right\} \]
\end{document}